\section{Sp�cification}
\label{sec:spec}

\subsection{L'Environnement}
\label{sec:environnement}
\paragraph{}L'environnement est une ville compos� de diff�rentes infrastructures:
\begin{itemize}
\item Trac� (routes)
\item Batiments:
\begin{itemize}
\item Maison (domicile des personnages)
\item Travail (lieux de travails des personnages)
\item Loisir
\end{itemize}
\end{itemize}

\paragraph{}Les diff�rents batiments auront:
\begin{itemize}
\item Un nom
\item Un nombre maximum d'utilisateurs
\item Une influence variable sur l'�motion des personnages (exemple: Un batiment Loisir rendra le personnage plus heureux, au contraire, aller travailler le rendra plus malheureux)
\item Une adresse
\item Un taille
\end{itemize}

\paragraph{}Chaque batiments poss�deront ses characteristiques propres:
\begin{itemize}
\item Maison
\begin{itemize}
\item Un ou plusieurs propri�taire
\end{itemize}
\item Travail et Loisir:
\begin{itemize}
\item Des horaires d'ouvertures et de fermetures
\item Un temps d'utilisation moyen
\item Une file d'attente
\end{itemize}
\end{itemize}

\paragraph{}Les trac�s (routes) permettront aux personages de se d�placer dans l'environnement. L'environnement sera donc un quadrillage de diff�rentes infrastructures, ainsi qu'une horloge permettant de simuler les diff�rentes phases d'une journ�e (nuit, matin, apr�s-midi, soir).

\subsection{Personnages}
\label{sec:personnages}
\paragraph{}Le r�el but du projet est d'analyser l'�volution d'une population dans un milieu urbain et cela sous la forme d'un jeu. Le point le plus important sera donc la vie quotidienne des personnes.

\paragraph{}Chaque personnages aura pour informations de base:
\begin{itemize}
\item Un nom
\item Un age
\item Un sexe
\item Une adresse (Un domicile represent� par un Batiments de type Maison)
\item Un travail, avec des horaires de travail
\end{itemize}

\paragraph{}Chaque personnages suivra une routine mod�lis� par une suite d'action, laquelle pourra �tre modifi� par le joueur. Les diff�rents type d'actions seront:
\begin{itemize}
\item Dormir
\item Se divertir
\item Aller travailler
\item Se d�placer
\end{itemize}

\paragraph{}Chaque personnages poss�dera trois jauge: �motion, Money et Family; repr�sentant respectivement son besoin en repos, travail, et divertisement. Chaque jauge diminuera au fur et � mesure du temps, le seul moyen de les augmenter est d'utiliser les batiments correspondant.

\subsection{Fonctionnalit�s}
\label{sec:fonctionnalite}
\paragraph{} Fonctionnalit�s du programme:
\begin{itemize}
\item Le joueur aura plusieurs actions possibles afn d'infuer sur l'�volution de la ville:
\begin{itemize}
\item Agir sur le temps, en stoppant l'horloge, mais aussi en l'accelerant ou la ralentissant
\item Acceder aux informations des batiments
\end{itemize}
\item Le joueur aura plusieurs actions possibles afn d'infuer sur l'�volution de la population:
\begin{itemize}
\item Acceder aux diff�rentes informations des personnages:
\begin{itemize}
\item Les informations de base (Nom, Age, Domicile, Travail)
\item Evolution du comportement du personnage dans le temps via des graphiques
\item La liste des actions � venir par le personnages
\end{itemize}
\item Agir sur le comportement des diff�rents personnages:
\begin{itemize}
\item Ajouter des actions � faire au personnage
\item Supprimer des actions pr�vu dans l'activit� du personnage
\end{itemize}
\end{itemize}
\end{itemize}