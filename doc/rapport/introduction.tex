\section{Introduction}
\label{sec:introduction}

\paragraph{Le projet}
Le projet consiste � la cr�ation d'un jeu vid�o simulant une vie urbaine, dans laquelle
plusieurs individus vivent leur vie au sein d'une ville. L'utilisateur pourra infuer sur le comportement
des individus et les param�tres de la ville.

\paragraph{Fonctionnalit�s} Fonctionnalit�s du programme:
Le joueur aura plusieurs actions possibles afn d'infuer sur l'�volution de la ville. Tout
d'abord il pourra agir sur le temps en l?acc�l�rant ou en le mettant en pause. Ensuite il pourra acc�der
� des informations sur les personnages comme : leurs informations de bases, leur historique ou leurs
objectifs directs (exemple : ce personnage se rend � la piscine). Ces informations pourront �tre
chang�es par l'utilisateur et il pourra ainsi renommer un personnage, le faire d�m�nager ou le faire
rentrer chez lui par exemple. De m�me pour les b�timents, l'utilisateur pourra acc�der � ses
informations et les modifer. Il pourra donc par exemple modifer les horaires d'ouverture d'un lieu ou
faire varier son nombre d'utilisateur maximum. Le joueur pourra donc modifer � sa guise ses
informations et voir ce que ces modifcations apportent de bon ou de mauvais sur � population de la
ville.

\paragraph{Nos motivations}
Nous avons choisi ce projet car il repr�sente une opportunit� pour chacun de nous
d?explorer des domaines/notions qui nous int�ressent, et dans lesquelles nous voulons nous
perfectionner.
